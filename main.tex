\documentclass{article}

%\usepackage{culmus}

\usepackage[T1]{fontenc}
\usepackage{ucs}   % package to add unicode support
\usepackage[utf8x]{inputenc}  % adding the UTF-8 encoding
\usepackage[hebrew,english]{babel} %      English is primary Hebrew is secondary
\usepackage{palatino} % Use the Palatino font by default
\setlength{\parindent}{0pt} % Disable paragraph indentation

\usepackage{verse, anyfontsize}
\usepackage{xcolor}
\usepackage{scalerel}
\usepackage{stackengine}

\renewcommand{\poemtitlefont}{\normalfont\bfseries\Huge\centering} % Define the poem title style
\setlength{\stanzaskip}{0.75\baselineskip} % The distance between stanzas

% For fancy initials
\usepackage{ebgaramond}
\usepackage{lettrine}


\begin{document}
	
	\poemtitle{Book 1}
	\settowidth{\versewidth}{Instruct me, for Thou know'st; Thou from the first}
	\poemlines{10}
	\input{content/book1}
	
	\newpage
	\setcounter{footnote}{0}
	\poemtitle{Book 2}
	\settowidth{\versewidth}{Instruct me, for Thou know'st; Thou from the first}
	\poemlines{10}
	\input{content/book2}

	\newpage
	\setcounter{footnote}{0}
	\poemtitle{Book 3}
	\settowidth{\versewidth}{Instruct me, for Thou know'st; Thou from the first}
	\poemlines{10}
	\input{content/book3}
	\newpage
	
	\newpage
	\setcounter{footnote}{0}
	\poemtitle{Book 4}
	\settowidth{\versewidth}{Instruct me, for Thou know'st; Thou from the first}
	\poemlines{10}
	\input{content/book4}
	\newpage
	
	\newpage
	\setcounter{footnote}{0}
	\poemtitle{Book 5}
	\settowidth{\versewidth}{Instruct me, for Thou know'st; Thou from the first}
	\poemlines{10}
	\input{content/book5}
	\newpage

	\newpage
	\setcounter{footnote}{0}
	\poemtitle{Book 6}
	\settowidth{\versewidth}{Instruct me, for Thou know'st; Thou from the first}
	\poemlines{10}
	\input{content/book6}
	\newpage
	
	\newpage
	\setcounter{footnote}{0}
	\poemtitle{Book 7}
	\settowidth{\versewidth}{Instruct me, for Thou know'st; Thou from the first}
	\poemlines{10}
	\input{content/book7}
	\newpage
	
	\newpage
	\setcounter{footnote}{0}
	\poemtitle{Book 8}
	\settowidth{\versewidth}{Instruct me, for Thou know'st; Thou from the first}
	\poemlines{10}
	\input{content/book8}
	\newpage
		
	\newpage
	\setcounter{footnote}{0}
	\poemtitle{Book 9}
	\settowidth{\versewidth}{Instruct me, for Thou know'st; Thou from the first}
	\poemlines{10}
	\input{content/book9}
	\newpage
	
	\newpage
	\setcounter{footnote}{0}
	\poemtitle{Book 10}
	\settowidth{\versewidth}{Instruct me, for Thou know'st; Thou from the first}
	\poemlines{10}
	\input{content/book10}
	\newpage
	
	\newpage
	\setcounter{footnote}{0}
	\poemtitle{Book 11}
	\settowidth{\versewidth}{Instruct me, for Thou know'st; Thou from the first}
	\poemlines{10}
	\input{content/book11}
	\newpage
	
	\newpage
	\setcounter{footnote}{0}
	\poemtitle{Book 12}
	\settowidth{\versewidth}{Instruct me, for Thou know'st; Thou from the first}
	\poemlines{10}
	\input{content/book12}
	\newpage
	
\end{document} 
